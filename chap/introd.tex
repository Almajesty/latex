\chapter*{Введение}
\addcontentsline{toc}{chapter}{Введение}

1. Введение
Лечение стриктуры мочеточника до сих пор остается сложной задачей для урологов. Для тех, кому хирургическая реконструкция противопоказана, для устранения обструкции верхнего отдела мочеточника обычно рекомендованы перкутанные нефростомические трубки (ПНТ) или двухпетлевые стенты [1]. Однако ПНТ могут вызывать эрозию кожи, инфекцию мочевыводящих путей, а также трубка может закупориться, что негативно влияет на качество жизни [1,2]. Несмотря на то, что вероятность успеха установки двухпетлевого стента высока, каждые 3-6 месяцев требуется замена стента, что приносит пациентам значительные неудобства и дополнительные затраты [3]. Таким образом, необходима новая процедура для устранения обструкции мочеточника и одновременного преодоления вышеупомянутых недостатков.

Чтобы решить эти задачи, для лечения стриктуры мочевыводящих путей стали использовать саморасширяющийся металлический мочеточниковый стент с покрытием (URS) (Allium, Allium LTD, Израиль) [3]. Этот тип стента доказал свою безопасность и эффективность в небольших исследованиях [3,4]. Однако ни в одном исследовании не сообщалось о его использовании на примере большой популяции, особенно среди китайских пациентов. Поэтому мы провели крупномасштабное одноцентровое проспективное исследование на базе большого центра для оценки безопасности и эффективности стентов Allium при стриктурах мочеточников, вызванных различной этиологией.



Корреспонденцию направлять автору: Отделение урологии, Институт урологии (лаборатория реконструктивной урологии), Западно-Китайская больница, Сычуаньский университет, № 37 Xue Xiang, Chengdu, Sichuan,, 610041, КНР.

Адрес электронной почты: weixinscu@scu.edu.cn (К. Вей). 
1 Xiaoshuai Gao, Turun Song и Liao Peng считаются соавторами.
Списки содержания доступны на сайте ScienceDirect
https://doi.org/10.1016/j.ijsu.2021.106161
Поступила 12 августа 2021 г.; Поступила в отредактированном виде 18 октября 2021 г.; Принята 28 октября 2021 г.
Доступна онлайн 30 октября 2021 г.
1743-9191/© 2021 IJS Publishing Group Ltd. Опубликовано Elsevier Ltd. Все права защищены.
