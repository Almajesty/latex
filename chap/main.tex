\chapter{2. Метод}
\section{2.1. Исследуемая популяция и данные}

Это исследование было опубликовано в соответствии с критериями STROCSS [5]. Критериями включения были пациенты с клинически диагностированной стриктурой мочеточника и возрастом старше 14 лет. Критериями исключения считаются пациенты с выраженной стриктурой уретры, затрудненным введением эндоскопа и невозможностью проведения операции; неконтролируемое острое и хроническое воспаление мочеполовой системы; тяжелая гематурия, которая может затруднить визуализацию сектора обзора при эндоскопии; беременные женщины или женщины в период менструации; пациенты с тяжелыми системными заболеваниями, которым противопоказаны анестезия или хирургическое вмешательство.
Все пациенты были проинформированы о рисках хирургического вмешательства, было получено их подписанное согласие. С января 2019 г. по июль 2020 г. 150 пациентам со стриктурой мочеточника в нашей больнице установили URS Allium.
Три пациента не стали проходить последующее наблюдение, и в итоге в это исследование вошло 147 пациентов. Были собраны демографические характеристики, оперативные параметры, осложнения и исходы, включая пол, возраст, индекс массы тела, сторону, этиологию, место и длину стриктуры, объем гидронефроза, норму СКФ (скорость клубочковой фильтрации) пораженной почки, уровни креатинина сыворотки и азота мочевины, время операции, осложнения, продолжительность пребывания в больнице, связанные со стентами симптомы и расходы на госпитализацию. Объем гидронефроза рассчитывали на основании данных компьютерной томографии (КТ): объем гидронефроза = длина * ширина * глубина * 0,523 [6].

\section{2.2. Хирургическая техника}

Все операции выполнял один и тот же опытный уролог. Установка URS проводилась в положении для литотомии под общей анестезией. В мочевой пузырь ввели жесткий цистоскоп, а в закупоренный мочеточник ретроградно ввели проводник. Расположение и длину стриктуры мочеточника определяли под рентгеноскопическим контролем с помощью ретроградной либо антероградной рентгенографии (если у пациента была нефростомическая трубка) или с помощью обеих сразу. Затем в место обструкции устанавливали баллонный катетер длиной 6 см для дилатации мочеточника и стриктуру расширяли при давлении до 25 атм в течение 3 мин.
Если стриктура была длиннее 6 см, ее расширяли сверху вниз несколько раз. После подтверждения на УЗИ, что суженный сегмент был удовлетворительно расширен, по направляющему проводнику вводили металлический мочеточниковый стент с покрытием размера 8F/10F. Затем в суженный мочеточник под флюороскопией пропускали URS с металлическим покрытием. После удовлетворительного высвобождения стента снова выполняли рентгенографию для подтверждения положения стента и проходимости мочеточника. У пациентов с атрезией мочеточника рубцовую ткань сегмента атрезии эндотомически исссекают с помощью гольмиевого лазера под контролем уретероскопа, а затем ретроградно вводят проводник. Затем осуществляется дилатация мочеточника и стент устанавливается так, как описано выше.
2.3. Протокол последующего наблюдения
Рутинное наблюдение включало анализ крови, анализ мочи, КТ брюшной полости, определение креатинина сыворотки и азота

переменных до и после операции использовали парный Т-критерий. Для выявления факторов риска хирургической неудачи был использован бинарный логистический регрессионный анализ. Когда одномерный анализ давал p < 0,05, выполняли многомерный анализ. Весь статистический анализ осуществлялся с использованием программного обеспечения SPSS версии 22.0, и p < 0,05 указывало на статистические различия.

\section{3. Результаты}

Сто сорок семь пациентов со 157 почечными единицами подверглись установке URS, 10 пациентов имели двусторонние стриктуры мочеточников, и 26 почечных единиц подверглись установке двух стентов в тандеме, поскольку стеноз мочеточника был слишком длинным. Основные характеристики пациентов подробно представлены в Таблице 1. Средний возраст пациентов составил 45,1 года, 94 (63,9\%) операции провели на мужчинах, у 59 (40,1\%) пациентов операция выполнялась на левом мочеточнике. Участок со стриктурой наиболее часто встречался в проксимальном отделе мочеточника (n = 69, 43,9\%), а средняя длина стриктуры мочеточника составляла 3,2 см.. 66 (42,0\%) стриктур мочеточников были вызваны камнями мочеточника или конкрементами после эндоскопического лечения, 24 (15,2\%) – открытой уретеропластикой, 13 (8,2\%) – послеоперационным или радиационным лечением по поводу рака, 11 (7,0\%) – вследствие травмы из-за урологической операции, 13 (8,2\%) случаев развились после трансплантации почки, 7 (4,4\%) стриктур возникли после операций по поводу доброкачественных заболеваний в области гинекологии и акушерства, 10 (6,3\%) рецидивирующих стриктур ранее лечились посредством внутренних разрезов или баллонной дилатации и 13 ( 8,2\%) стриктур мочеточников не имели определенной этиологии. У 43 (27,39\%) больных в сегменте стриктур имелась атрезия. У 25 (15,9\%) пациентов перед операцией была установлена двухпетлевая трубка, а у 67 (42,6\%) — дренирование осуществлялось посредством ПНТ.
Все стенты были установлены успешно. Характеристики периоперационного периода представлены в Таблице 2. Для 132 мочеточниках (84,1\%), чтобы установка металлических стентов прошла успешно, потребовалась баллонная дилатация. Двадцати шести пациентам (16,56\%) было установлено два стента в тандеме из-за слишком протяженного стеноза мочеточника, а всем остальным был установлен один стент. Среднее время операции составило 70,0 мин., а средняя стоимость госпитализации составила 11 119 долларов США. Наиболее частым осложнением была послеоперационная гематурия в 13 случаях (8,8\%), за ней следовали инфекция мочевыводящих путей - 11 (7,5\%) и боль - 8 (5,4\%). мочевины в первый месяц после операции и затем каждые 3 месяца. Однофотонная эмиссионная КТ (ОФЭКТ) повторно проводилась только в первый месяц для оценки почечной функции пораженной почки. Во время последующего наблюдения регистрировались связанные со стентами симптомы. Хирургическая неудача определяется как усиление гидронефроза или ухудшение почечной функции из-за миграции, окклюзии или инкрустации стентов.

\section{2.4. Статистический анализ}

Непрерывные переменные были выражены как среднее значение ± стандартное отклонение, а категориальные переменные были описаны как частота (пропорции). Для сравнения непрерывных

После медианы наблюдения в 15 месяцев (диапазон 9-20 месяцев) 115 (73,2\%) стентов были оставлены в теле пациентов. В числе связанных со стентом осложнений, у 10 (6,8\%) пациентов отмечалась устойчивая боль, у 6 (4,1\%) пациентов - устойчивая гематурия, у 8 (5,4\%) пациентов -симптомы нижних мочевыводящих путей и у 8 (5,4\%) - рецидивирующие инфекции мочевыводящих путей.
Общий показатель успешности этой процедуры составил 73,2\% (115/157). Было 42 неудачных случая, в том числе 37 (23,6\%) миграций стента, 2 (1,3\%) окклюзии и 3 (1,9\%) инкрустации камнями. В случаях миграции 30 стентов были эндоскопически приведены в нормальное положение и 7 стентов были заменены. 2 пациентам с окклюзией стента была выполнена замена стента. У пациентов с инкрустацией стентов камни были успешно удалены при помощи ретроградной гибкой литотрипсии мочеточников. После второй операции на последнем осмотре все стенты оставались на месте и сохраняли хорошую проходимость.
Результаты наблюдения представлены в Таблице 3. Результаты наблюдения через 9 мес. после операции следующие. Объем гидронефроза после операции значительно уменьшился с (67,9 ± 34,9) см3 до операции до (33,5 ± 49,8) см3. Уровень креатинина крови (103,0 ± 54,5 против 92,8 ± 45,1 мкмоль/л, P = 0,019) и азота мочевины (6,6 ± 6,7 против 5,2 ± 2,3 ммоль/л, P = 0,012) после операции также значительно снизились. Результаты последнего наблюдения соответствовали результатам через 9 мес. после операции. Объем гидронефроза (Р = 0,0001), уровень креатинина крови (Р = 0,034) и азота мочевины (Р = 0,032) после операции значительно уменьшились. Тем не менее, на последнем контроле не было обнаружено существенных изменений в СКФ пораженной почки (25,0 ± 16,1 против 23,3 ± 

Предиктор неудачи процедуры был проанализирован и продемонстрирован в Таблице 4, и было обнаружено, что независимым фактором риска несостоятельности стента была обструкция дистального отдела мочеточника (ОР 1,77, 95\% ДИ 1,15, 2,73, P = 0,009). Других независимых факторов, влияющих на неудачу операции, не обнаружено. Длина, этиология и предшествовавшая операция по поводу стриктуры мочеточника влияют на вероятность успеха уретеропластики, но эти факторы не снижают вероятность успеха использования URS Allium.
4. Обсуждение
Насколько нам известно, это крупнейшее проспективное исследование применения стента Allium при стриктуре мочеточника. Наше исследование показало, что URS эффективно устраняет обструкцию мочеточника. Общий показатель успешного хирургического лечения составил 73,2\% при последующем наблюдении в течение 15 месяцев. Даже в случаях неудачи после лечебных мероприятий все стенты находились на месте и поддерживали хорошую проходимость. Кроме того, методика безопасна и хорошо переносится, ни одному пациенту не потребовалось удаление стента из-за осложнений.
Стриктура мочеточника является распространенным заболеванием и остается большой проблемой для урологов [7]. В настоящее время существует множество вариантов лечения стриктур мочеточника, включая открытую реконструктивную хирургию, эндоуретеротомию, баллонную дилатацию, стентирование мочеточника и ПНТ [8]. Лечение следует подбирать в зависимости от этиологии, тяжести стриктуры мочеточника, осложнений и прогноза [9]. ПНТ наименее предпочтительна из-за наружного типа дренирования, ограничивающего активность пациента и отрицательно влияющего на качество жизни пациента [10]. Кроме того, ПНТ требуют частой замены [3]. Введение после баллонной дилатации двухпетлевой трубки стало рутинным методом лечения стриктур мочеточника [11]. К сожалению, частота отказов традиционных двухпетлевых трубок составляет 30–45\% [12,13]. Cвязанные со стентированием cимптомы и необходимость частых замен также значительно снижают качество жизни [1]. Металлический стент разработан так, чтобы избежать частой замены, которую требуют  двухпетлевой трубки, но имеется высокий риск миграции, прорастания тканей и инкрустации стентов [7,14].
URS Allium представляет собой полностью покрытый металлический стент большого калибра, который является интересной альтернативой для долгосрочного дренирования мочеточника, поскольку он предотвращает нарушение дренажа из-за врастания тканей [3]. Однако сообщалось о миграции стента как о распространенном осложнении [15]. Исследования показали, что URS Allium обеспечивает хороший дренаж при доброкачественных и злокачественных стриктурах мочеточников [3,16–18]. Московиц и соавт. сообщили о своем 6-летнем опыте применения URS Allium в лечении 49 случаях стриктур мочеточников [3].
В течение среднего периода наблюдения сроком 17 месяцев только 7 (14,2\%) стентов мигрировали и были удалены [3]. Исследование также показало, что у 20\% пациентов проходимость мочеточников восстановилась после удаления стента, и вмешательство не потребовалось [3]. Кроме того, в 2015 г. в многоцентровом исследовании сообщалось о 107 стриктурах мочеточников у 92 пациентов [18]. Миграция стента произошла у 10,7\% пациентов в течение восьми месяцев пребывания стента в теле, и только 1 стент оказался заблокирован в течение среднего периода наблюдения 27 13,1 мл/мин./1,73 м2, P = 0,051) и в поглощении пораженной почкой (31,7 ± 15,6 против 30,9 ± 16,4, P = 0,400) по сравнению с исходным уровнем.

месяцев [18]. Гуандалино и соавт. обнаружили, что частота миграции стента составила 18,9\%, а факторами риска миграции были возраст старше 60 лет и женский пол. Семь (18,9\%) стентов были удалены из-за инфекции и непереносимости [4].
В нашей исследовательской группе все стенты устанавливались ретроградно. Для успешного применения URS 84,1\% пациентам потребовалась баллонная дилатация. В ходе наблюдения было заменено 9 стентов, а 30 мигрировавших стентов эндоскопически привели в нормальное положение. Далее все стенты сохраняли проходимость вплоть до последнего контрольного осмотра. Мы также определили, что стриктура дистального отдела мочеточника является независимым фактором риска несостоятельности стента. Это было подтверждено исследованием Голдсмита о том, что дистальная обструкция увеличивает риск отказа стента [19]. Таким образом, в лечении стриктур дистального отдела мочеточника при установке металлического стента имеется риск продвижения стента в мочевой пузырь. Кроме того, как факторы риска отказа стента также были зарегистрированы радиационная терапия и инфекция мочевыводящих путей в анамнезе [20,21].

В предыдущих исследованиях сообщалось о нескольких связанных с URS Allium осложнениях, включая гематурию, инфекции мочевыводящих путей, раздражение и образование инкрустации [3,4,16–18]. Эти осложнения также имели место в нашей когорте. Исследования показали, что правильный размер стента может уменьшить число осложнений [22], поэтому в нашем исследовании мы индивидуально подбирали подходящую модель стента в соответствии с особенностями каждого пациента. В нашей когорте общая частота периоперационных осложнений составила 21,7\%, включая гематурию, инфекцию мочевыводящих путей и боль. При этом общее количество связанных со стентированием осложнений составило всего 15,0\%, и все они переносились хорошо.
По сравнению с двухпетлевыми трубками, URS Allium имеет более высокую частоту миграции, но более низкую частоту инкрустации или окклюзии [23]. Независимо от причины стриктуры мочеточника, своевременное устранение обструкции может защитить функцию почек [8,24]. Обычные полимерные стенты эффективны, но их необходимо регулярно менять каждые 3–6 мес., часто развивается гематурия, раздражение мочевого пузыря или инфекции мочевыводящих путей [25]. Чен и соавт., сравнив безопасность и эффективность обычных и металлических стентов, обнаружили, что металлические стенты имели более высокие показатели проходимости через 6 месяцев (100\% против 83,8\%) и 1 год (91,7\% против 40,0\%), чем обычные стенты [23]. Следует отметить, что общая частота осложнений металлических стентов была ниже, чем обычных стентов (36,7\% против 63,6\%), а показатель качества жизни, наоборот, был выше (30,9 ± 2,8 против 23,6 ± 1,8) [23].
В нашем исследовании почти все пациенты сохраняли эффективный дренаж непрерывно в течение 15 месяцев или после корректировки положения стента, а средняя общая стоимость лечения составила 11 119 долларов США. Хектор и соавт. сообщили, что стоимость их металлического стента Resonance для лечения стриктуры мочеточника аналогична нашей, а годовая стоимость составляет 13 633 доллара США [7]. Кроме того, авторы обнаружили, что хотя металлические стенты дороже, чем полимерные, использование металлических стентов связано со снижением годовой стоимости на 43\% (13 633 долл. США против 23 999

долл. США) [7]. Эта цифра не учитывает другой экономический фактор, наравне с частой заменой стента, а именно пропуски работы из-за визитов в больницу.

5. Выводы

URS безопасен и эффективен в терапии стриктур мочеточников, имеет ограниченный риск осложнений и хорошие долгосрочные результаты. Для пациентов, которым не подходит открытая хирургическая реконструкция, вариантом лечения является URS.

Происхождение и экспертная оценка

Не по заказу, внешняя экспертная оценка.

Этическое одобрение

Получено одобрение Комитета по этики Западно-Китайской больницы, регистрационный номер одобрения 2019-009.

Источники финансирования

Это исследование получило поддержку новой клинической технологии Западно-Китайской больницы Сычуаньского университета (грант № 20HXJS002).

Авторский вклад

Сяошуай Гао: Написание первоначального проекта. Турун Сонг: концептуализация, методология. Ляо Пэн: Расследование. Чи Юань: Программное обеспечение. Вэй Ван: Обзор. Цзисян Чен: Редактирование. Кайвен Сяо: Курирование данных. Синь Вэй: Надзор.

Регистрационный уникальный идентификационный номер (UIN)

Название реестра: Реестр исследований.
Уникальный идентификационный номер или регистрационный идентификатор: researchregistry7266.
Гиперссылка на вашу конкретную регистрацию (должна быть общедоступной и будет проверяться): https://www.researchregistry.com/browse-the-registry#home/

Гарант

Xin Wei

Заявление об авторском вкладе CRediT

Xiaoshuai Gao: Текст – первоначальный вариант. Chi Yuan: концептуализация, методология. Liao Peng: Расследование. Chi Yuan: Программное обеспечение. Wei Wang: Текст – рецензирование и редактирование. Jixiang Chen: Текст – обзор и редактирование. Kaiwen Xiao: Курирование данных. Xin Wei: Надзор.

Заявление о конфликте интересов

Авторы заявили, что у них нет конфликтов интересов, связанных с настоящей работой.

Приложение А. Дополнительные данные

Дополнительные данные к этой статье можно найти в Интернете по адресу https://doi. org/10.1016/j.ijsu.2021.106161.

Использованная литература

advanced malignancies: are there differences? Urology 64 (2004) 895–899, https://doi.org/10.1016/j.urology.2004.06.029. [10] K. Pavlovic, D. Lange, B.H. Chew, Stents for malignant ureteral obstruction, Asian J Urol 3 (2016) 142–149, https://doi.org/10.1016/j.ajur.2016.04.002. [11] M. Kim, G. Song, S.H. Park, M. Sohn, S.H. Song, H.K. Park, et al., Outcomes of patients with ureteral obstruction who achieved stent-free state following balloon dilatation, Scand J Urol 50 (2016) 396–400, https://doi.org/10.1080/ 21681805.2016.1209690. [12] O. Yossepowitch, D.A. Lifshitz, Y. Dekel, M. Gross, D.M. Keidar, M. Neuman, et al., Predicting the success of retrograde stenting for managing ureteral obstruction, J. Urol. 166 (2001) 1746–1749. [13] A.M. Ganatra, K.R. Loughlin, The management of malignant ureteral obstruction treated with ureteral stents, J. Urol. 174 (2005) 2125–2128, https://doi.org/ 10.1097/01.ju.0000181807.56114.b7. [14] S.E. Elsamra, D.A. Leavitt, H.A. Motato, J.I. Friedlander, M. Siev, M. Keheila, et al., Stenting for malignant ureteral obstruction: tandem, metal or metal-mesh stents, Int. J. Urol. 22 (2015) 629–636, https://doi.org/10.1111/iju.12795. [15] G.A. Barbalias, E.N. Liatsikos, C. Kalogeropoulou, D. Karnabatidis, P. Zabakis, A. Athanasopoulos, et al., Externally coated ureteral metallic stents: an unfavorable clinical experience, Eur. Urol. 42 (2002) 276–280, https://doi.org/10.1016/s0302- 2838(02)00281-6. [16] Z. Bahouth, B. Moskovitz, S. Halachmi, O. Nativ Allium Stents, A novel solution for the management of upper and lower urinary tract strictures, Rambam Maimonides Med J 8 (2017), https://doi.org/10.5041/rmmj.10313. [17] C. Leonardo, M. Salvitti, G. Franco, C. De Nunzio, G. Tuderti, L. Misuraca, et al., Allium stent for treatment of ureteral stenosis, Minerva Urol. Nefrol. 65 (2013) 277–283. [18] Z. Bahouth, G. Meyer, S. Halachmi, O. Nativ, B. Moskowitz, Multicenter experience with Allium ureteral stent for the treatment of ureteral stricture and fistula, Harefuah 154 (2015) 753–756, 806. [19] Z.G. Goldsmith, A.J. Wang, L.L. Ba˜nez, M.E. Lipkin, M.N. Ferrandino, G. M. Preminger, et al., Outcomes of metallic stents for malignant ureteral obstruction, J. Urol. 188 (2012) 851–855, https://doi.org/10.1016/j. juro.2012.04.113. [20] H.J. Wang, T.Y. Lee, H.L. Luo, C.H. Chen, Y.C. Shen, Y.C. Chuang, et al., Application of resonance metallic stents for ureteral obstruction, BJU Int. 108 (2011) 428–432, https://doi.org/10.1111/j.1464-410X.2010.09842.x. [21] J.A. Brown, C.L. Powell, K.R. Carlson, Metallic full-length ureteral stents: does urinary tract infection cause obstruction? ScientificWorldJournal 10 (2010) 1566–1573, https://doi.org/10.1100/tsw.2010.162. [22] U. Nagele, M.A. Kuczyk, M.

[1] E. Liatsikos, P. Kallidonis, I. Kyriazis, C. Constantinidis, K. Hendlin, J. U. Stolzenburg, et al., Ureteral obstruction: is the full metallic double-pigtail stent the way to go? Eur. Urol. 57 (2010) 480–486, https://doi.org/10.1016/j. eururo.2009.02.004. [2] T. Kljuˇcevˇsek, V. Pirnovar, D. Kljuˇcevˇsek, Percutaneous nephrostomy in the neonatal period: indications, complications, and outcome-A single centre experience, Cardiovasc. Intervent. Radiol. 43 (2020) 1323–1328, https://doi.org/ 10.1007/s00270-020-02528-z. [3] B. Moskovitz, S. Halachmi, O. Nativ, A new self-expanding, large-caliber ureteral stent: results of a multicenter experience, J. Endourol. 26 (2012) 1523–1527, https://doi.org/10.1089/end.2012.0279. [4] M. Guandalino, S. Droupy, A. Ruffion, G. Fiard, M. Hutin, D. Poncet, et al., The Allium ureteral stent in the management of ureteral stenoses, a retrospective, multicenter study, Prog. Urol. 27 (2017) 26–32, https://doi.org/10.1016/j. purol.2016.11.005. [5] R. Agha, A. Abdall-Razak, E. Crossley, N. Dowlut, C. Iosifidis, G. Mathew Strocss, Guideline: strengthening the reporting of cohort studies in surgery, Int. J. Surg. 72 (2019) 156–165, https://doi.org/10.1016/j.ijsu.2019.11.002, 2019. [6] V.Y. Leung, D.D. Rasalkar, J.X. Liu, B. Sreedhar, C.K. Yeung, W.C. Chu, Dynamic ultrasound study on urinary bladder in infants with antenatally detected fetal hydronephrosis, Pediatr. Res. 67 (2010) 440–443, https://doi.org/10.1203/ PDR.0b013e3181d22b91. [7] H.L. L´opez-Huertas, A.J. Polcari, A. Acosta-Miranda, T.M. Turk, Metallic ureteral stents: a cost-effective method of managing benign upper tract obstruction, J. Endourol. 24 (2010) 483–485, https://doi.org/10.1089/end.2009.0192. [8] C.C. Li, J.R. Li, L.H. Huang, S.W. Hung, C.K. Yang, S.S. Wang, et al., Metallic stent in the treatment of ureteral obstruction: experience of single institute, J. Chin. Med. Assoc. 74 (2011) 460–463, https://doi.org/10.1016/j.jcma.2011.08.017. [9] J.H. Ku, S.W. Lee, H.G. Jeon, H.H. Kim, S.J. Oh, Percutaneous nephrostomy versus indwelling ureteral stents in the management of extrinsic ureteral obstruction in

Horstmann, J. Hennenlotter, K.D. Sievert, D. Schilling, et al., Initial clinical experience with full-length metal ureteral stents for obstructive ureteral stenosis, World J. Urol. 26 (2008) 257–262, https://doi.org/ 10.1007/s00345-008-0245-4. [23] Y. Chen, C.Y. Liu, Z.H. Zhang, Malignant ureteral obstruction: experience and comparative analysis of metallic versus ordinary polymer ureteral stents, World J. Surg. Oncol. 17 (2019) 74, https://doi.org/10.1186/s12957-019-1608-6. [24] A.P. Modi, C.R. Ritch, D. Arend, R.M. Walsh, M. Ordonez, J. Landman, et al., Multicenter experience with metallic ureteral stents for malignant and chronic benign ureteral obstruction, J. Endourol. 24 (2010) 1189–1193, https://doi.org/ 10.1089/end.2010.0121. [25] M. Duvdevani, B.H. Chew, J.D. Denstedt, Minimizing symptoms in patients with ureteric stents, Curr. Opin. Urol. 16 (2006) 77–82, https://doi.org/10.1097/01. mou.0000193375.29942.0f.

% \chapter{Second Chapter}
% \section{First Section}
% \lipsum[20-30]
% \section{Second Section}
% \lipsum[30-50]